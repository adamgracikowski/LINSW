\documentclass{article}
%----------------------------------------------------------------------
\usepackage{graphicx}
\usepackage[T1]{fontenc}
\usepackage[polish]{babel}
\usepackage[utf8]{inputenc}
\usepackage{amsmath}
\usepackage{multirow}
\usepackage{geometry}
\usepackage{float}
\usepackage{siunitx}
\usepackage{enumitem}
\usepackage{listings}  % Pakiet do formatowania kodu
\usepackage{xcolor}    % Pakiet do kolorowania składni

% Definicja stylu dla Lua
\lstdefinelanguage{Lua}{
    morekeywords={and, break, do, else, elseif, end, false, for, function, if, in, local, nil, not, or, repeat, return, then, true, until, while},
    sensitive=true,
    morecomment=[l]--,
    morecomment=[s]{--[[}{]]},
    morestring=[b]",
    morestring=[b]',
}

% Definicja stylu dla msmtprc
\lstdefinelanguage{msmtprc}{
    morekeywords={account, host, port, from, auth, user, password, tls, tls_starttls, logfile},
    sensitive=false,
    morecomment=[l]\#,
}

\lstset{
    basicstyle=\ttfamily\small,
    keywordstyle=\color{blue}\bfseries,
    stringstyle=\color{red},
    commentstyle=\color{gray},
    backgroundcolor=\color{lightgray!20},
    frame=single,
    numbers=left,
    numberstyle=\tiny\color{gray},
    stepnumber=1,
    breaklines=true,
    captionpos=b
}
%----------------------------------------------------------------------
% strona tytułowa:

\begin{document}
\begin{titlepage}
    \centering
    \vfill    
    {\fontsize{40}{20}\selectfont \textbf{Linux w systemach wbudowanych} \par}
    \vspace{2cm}
    {\fontsize{30}{20}\selectfont Laboratorium L1 G1\par}
    \vspace{2cm}
    {\fontsize{20}{20}\selectfont Sprawozdanie\par}
    \vfill
    {\fontsize{15}{20}\selectfont Adam Grącikowski, 327350\par}
    \vspace{1cm}
    Warszawa, \today
\end{titlepage}


%----------------------------------------------------------------------

\section{Cel ćwiczenia laboratoryjnego}
\label{Cel}

 Celem ćwiczenia jest zapoznanie się z:

 \begin{itemize}
     \item środowiskiem Buildroot,
     \item platformą Raspberry Pi 4,
     \item akcesoriami dostępnymi na laboratorium,
 \end{itemize}

a także wygenerowanie dla Raspberry Pi 4 za pomocą Buildroota obraz systemu, wykorzystującego jądro Linux, spełniający wymagania opisane w rozdziale \ref{Wymagania}.

\section{Wymagania}
\label{Wymagania}

\begin{enumerate}[label=\arabic*.]
    \item System powinien używać \texttt{Initramfs} (początkowy ramdysk) jako głównego systemu plików.
    \item Powinno być ustawione hasło dla użytkownika  \texttt{root}.
    \item System powinien być wyposażony w serwer ssh (zalecany  \texttt{dropbear} - względnie łatwy w konfig
    uracji). Powinna istnieć możliwość zalogowania się do systemu z sieci.
    \item Standardowy komunikat  \texttt{“Welcome to Buildroot”} wyświetlany przy starcie systemu należy zastąpić komunikatem dostosowanym do indywidualnych potrzeb, zawierającym imię i nazwisko studenta,
    w moim przypadku \texttt{“Welcome Adam Gracikowski”}.
    \item Data i czas w systemie powinny być ustawiane automatycznie z serwera NTP.
    \item  System powinien zawierać skrypt napisany w języku Lua, który podczas uruchamiania systemu i jego wyłączania, skrypt wysyła e-maila z powiadomieniem o tym
    fakcie.
\end{enumerate}

\section{Opis modyfikacji i konfiguracji Buildroota}

\begin{enumerate}[label=\arabic*.]
    \item Pobranie środowiska Buildroot, rozpakowanie oraz dokonanie wstępnej konfiguracji poprzez: \texttt{make raspberrypi4\_64\_defconfig}.
    \item \texttt{System configuration > Run a getty (login prompt) after boot > TTY port} ustawione na \texttt{console}.
    \item \texttt{Build options > Mirrors and Download locations > Primary download site} ustawione na \texttt{http://192.168.137.24/dl}.
    \item \texttt{Toolchain > Toolchain type} ustawione na \texttt{External toolchain}.
    \item \texttt{Build options > Enable compiler cache} zaznaczone i \texttt{Compiler cache location} ustawione na \texttt{/home/adam/linsw/ccache-br}.
    \item \texttt{Filesystem images} zaznaczona opcja \texttt{initial RAM filesystem 
linked into linux kernel} oraz \texttt{Compression method} ustawione na \texttt{gzip}.
    \item W pliku \texttt{board/raspberrypi4-64/genimage.cfg.in} parametr \texttt{size} ustawiony na \texttt{128M}.
    \item \texttt{System configuration > Enable root login and password > Root password} ustawione na przykładowe hasło \texttt{root,123}.
    \item \texttt{Target packages > Networking applications} zaznaczona opcja \texttt{dropbear}.
    \item \texttt{System configuration > System banner} ustawione na \texttt{Welcome Adam Gracikowski}.
    \item \texttt{Target packages > Networking applications} zaznaczone \texttt{ntp} i dalej 
    \texttt{ntpd}, \texttt{ntpdate} oraz \texttt{ntptime}.
    \item \texttt{System configuration > Install timezone info > default local time} ustawione na \texttt{Europe/Warsaw}.
    \item \texttt{System configuration > Root filesystem overlay directories} ustawione na \texttt{board/overlay}.
    \item \texttt{Target packages > Interpreter languages and scripting} zaznaczone \texttt{lua} i dalej \texttt{Lua 5.3.x}.
    \item W folderze \texttt{board} utworzono folder \texttt{overlay}, a w nim plik \texttt{etc/init.d/S99email} i nadano mu uprawnienia do wykonywania przy pomocy polecenia: \texttt{chmod +x S99email}. Zawartość skryptu w języku Lua umieszczono w archiwum wraz ze sprawozdaniem.
    \item \texttt{Target packages > Mail} zaznaczone \texttt{heirloom-mailx} oraz \texttt{msmtp}.
    \item \texttt{Target packages > Libraries > Crypto} zaznaczone \texttt{CA Certificates} oraz \texttt{openssl support}.
    \item W folderze \texttt{board/overlay/etc} dodano pliki konfiguracyjne \texttt{msmtprc} oraz \texttt{mail.rc}. Prawa dostępu do pliku zostały ograniczone przy pomocy polecenia: \texttt{chmod 600 msmtprc}. Zawartość plików umieszczono w archiwum wraz ze sprawozdaniem. 
\end{enumerate}

\section{Opis plików konfiguracyjnych i skryptów dołączonych do archiwum}

Plik \texttt{msmtprc} to plik konfiguracyjny, dla programu \texttt{msmtp}, który służy do wysyłania e-maili z linii poleceń. W pliku m.in. zdefiniowane jest konto, które będzie używane do wysyłania wiadomości e-mail. W celu wykonania zadania stworzone zostało dedykowane konto \textit{Google} wraz z wygenerowanym dla niego \textit{App password}, które pozwala na automatyzacje wysyłania wiadomosci e-mailowych. Określony został rownież serwer i port.

Plik \texttt{mail.rc} określa, że w przypadku używania komend do wysyłania maili, system powinien używać programu \texttt{msmtp}.

Skrypt \texttt{S99email} to skrypt w jezyku Lua, który wysyła odpowiedni e-mail w zależnosci od akcji podanej jako parametr z konsoli. Polecenie przekazywane do konsoli jest również drukowane na konsoli w celu sprawdzenia działania skryptu. Skrypt ten jest automatycznie wywoływany podczas takich akcji jak \texttt{start}, \texttt{stop}, \texttt{reload} i \texttt{restart}. W skrypcie tym został wpisany na stałe w zmiennej \texttt{email\_recipient} adres e-mail adresata wiadomości (w celach testowych). W przypadku własnego testowania należy zmienić adres na własny.

\newpage

\subsection{Skrypt w języku Lua}

Poniżej znajduje się wykorzystany skrypt w języku Lua:

\begin{lstlisting}[language=Lua, caption=Skrypt Lua., label=lst:lua]
#!/usr/bin/lua

---
--- This script sends an email on system startup and shutdown with a delay.
---

local email_recipient = "adgrac@op.pl"
local subject_prefix = "System notification"

local function send_email(action)
    local subject = subject_prefix .. ": " .. action
    local body = "System " .. action

    -- Construct the msmtp command
    local email_command = string.format(
        "echo -e \"Subject: %s\\n\\n%s\" | msmtp %s",
        subject, body, email_recipient
    )

    -- Print execution status
    print("Executing Lua script at system " .. action .. "...")

    -- Introduce a 5-second delay before sending the email
    print("Waiting for 5 seconds before sending email...")
    os.execute("sleep 5")

    -- Print the command for debugging
    print("Executing command: " .. email_command)

    -- Execute the command
    os.execute(email_command)

    -- Confirm execution
    print("Email sent successfully for system " .. action)
end

local action = arg[1]

if action == "start" or action == "stop" then
    send_email(action)
else
    print("Usage: " .. arg[0] .. " <start|stop>")
end


\end{lstlisting}

\newpage

\subsection{Plik konfiguracyjny \texttt{msmtp}}

Poniżej znajduje się konfiguracja pliku \texttt{msmtprc}:

\begin{lstlisting}[language=msmtprc, caption=Plik konfiguracyjny \texttt{msmtp}., label=lst:msmtprc]
defaults
auth           on
tls            on
tls_starttls   off

account        gmail
host           smtp.gmail.com
port           465

from           linsw.mini@gmail.com
user           linsw.mini
password       ...

account default : gmail
\end{lstlisting}

\subsection{Plik konfiguracyjny \texttt{mail.rc}}

Poniżej znajduje się konfiguracja pliku \texttt{mail.rc}:

\begin{lstlisting}[language=msmtprc, caption=Plik konfiguracyjny \texttt{mail.rc}., label=lst:mailrc]
set sendmail=/usr/bin/msmtp
\end{lstlisting}

%----------------------------------------------------------------------
% bibliografia:

\begin{thebibliography}{9}
\bibitem{msmtp} \texttt{msmtp} ArchWiki - https://wiki.archlinux.org/title/Msmtp, dostęp \today
\bibitem{Arnaud R} Arnaud R, \textit{Send emails from your terminal with msmtp} - https://arnaudr.io/2020/08/24/send-emails-from-your-terminal-with-msmtp/, dostęp \today
\bibitem{app-password} App password - https://support.google.com/accounts/answer/185833?hl=en, dostęp \today
\end{thebibliography}
%----------------------------------------------------------------------
\end{document}