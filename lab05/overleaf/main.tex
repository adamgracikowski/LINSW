\documentclass{article}
%----------------------------------------------------------------------
\usepackage{graphicx}
\usepackage[T1]{fontenc}
\usepackage[polish]{babel}
\usepackage[utf8]{inputenc}
\usepackage{amsmath}
\usepackage{multirow}
\usepackage{geometry}
\usepackage{float}
\usepackage{siunitx}
\usepackage{enumitem}
\usepackage{dirtree}
\usepackage{verbatim}
\usepackage[colorlinks=true, linkcolor=blue, urlcolor=blue, citecolor=blue]{hyperref}

%----------------------------------------------------------------------
% strona tytułowa:

\begin{document}
\begin{titlepage}
    \centering
    \vfill    
    {\fontsize{40}{20}\selectfont \textbf{Linux w systemach wbudowanych} \par}
    \vspace{2cm}
    {\fontsize{30}{20}\selectfont Laboratorium L5 G1\par}
    \vspace{2cm}
    {\fontsize{20}{20}\selectfont Sprawozdanie\par}
    \vfill
    {\fontsize{10}{20}\selectfont Adam Grącikowski, 327350\par}
    \vspace{1cm}
    Warszawa, \today
\end{titlepage}


%----------------------------------------------------------------------
% spis treści:

\tableofcontents
\newpage
%----------------------------------------------------------------------
\section{Cel ćwiczenia laboratoryjnego}

Celem ćwiczenia laboratoryjnego było uruchomienie i dostosowanie rozwiązania zrealizowanego w ramach czwartego zadania laboratoryjnego, tym razem w środowisku opartym na systemie OpenWrt. W odróżnieniu od poprzedniego ćwiczenia, w którym wykorzystano środowisko Buildroot do budowy systemu, w niniejszym zadaniu zastosowano prekompilowany obraz OpenWrt. 

Ćwiczenie miało na celu zapoznanie się z konfiguracją i dostosowywaniem gotowego systemu OpenWrt do potrzeb konkretnej aplikacji.

\section{System Kolejkowania Zleceń Produkcyjnych}

Aplikacja, którą należało uruchomić w ramach niniejszego zadania, to program zaimplementowany podczas czwartego ćwiczenia laboratoryjnego. W moim przypadku była to aplikacja realizująca System Kolejkowania Zleceń Produkcyjnych.

Szczegółowe założenia projektowe oraz opis działania interfejsów — zarówno planisty, jak i pracownika linii produkcyjnej — zostały zawarte w raporcie do poprzedniego zadania laboratoryjnego.

\section{Przeniesienie rozwiązania na OpenWrt}

\subsection{Pobranie wersji prekompilowanego}

Pierwszym krokiem w kierunku przeniesienia rozwiązania na OpenWrt było pobranie i przeniesienie na płytkę Raspberry Pi 4 prekompilowanej wersji systemu.

Zgodnie z materiałami wykładowymi, jako źródło prekompilowanej wersji wybrana została odpowiednia dla platformy docelowej zakładka oficjalnej strony OpenWrt, czyli \href{https://downloads.openwrt.org/releases/24.10.1/targets/bcm27xx/bcm2711/}{brcm27xx/bcm2711}. 

Na tej stronie pobrany został plik skompresowany \texttt{rpi-4-ext4-factory.img.gz}. Plik ten został następnie rozpakowany, a obraz systemu został wgrany na kartę pamięci przy pomocy polecenia \texttt{dd}.

\subsection{Konfiguracja sieci w OpenWrt}

Zgodnie z materiałami wykładowymi, dokonana została modyfikacja obsługi sieci w OpenWrt przez podłączenie do rzeczywistej sieci.

Należało dokonać edycji zbioró \texttt{/etc/config/network} poprzez zamianę istniejącej konfiguracji interfejsu \texttt{lan} następującą konfiguracją:

\begin{verbatim}
config interface 'lan'
    option ifname 'eth0'
    option proto 'dhcp'
\end{verbatim}

Po skończonej edycji wywołano \texttt{/etc/init.d/network reload} oraz usunięto zbiór uruchamiający serwer DNS/DHCP dnsmasq poleceniem \texttt{rm /etc/rc.d/S19dnsmasq}.

\subsection{Konfiguracja środowiska OpenWrt}
\label{konfiguracja}

W celu uruchomienia aplikacji Systemu Kolejkowania Zleceń Produkcyjnych w środowisku OpenWrt, konieczne było przygotowanie systemu poprzez instalację odpowiednich pakietów. Ponieważ aplikacja została zaimplementowana w języku Python, pierwszym krokiem było zaktualizowanie listy dostępnych pakietów oraz zainstalowanie interpretera Python 3:

\begin{verbatim}
opkg update
opkg install python3
python3 --version
\end{verbatim}

Następnie zainstalowano biblioteki niezbędne do działania aplikacji, w szczególności związane z obsługą serwera HTTP oraz komunikacją WebSocket. Użyto do tego poniższego polecenia:

\begin{verbatim}
opkg install python3-tornado \
             python3-websocket-client \
             python3-websockets \
             python3-pyopenssl
\end{verbatim}

W dalszej kolejności zainstalowano menedżer pakietów \texttt{pip}, który umożliwia instalację dodatkowych modułów Python:

\begin{verbatim}
opkg install python3-pip
pip --version
\end{verbatim}

Na zakończenie, za pomocą \texttt{pip}, zainstalowano bibliotekę \texttt{UniversalGPIO}, która była wymagana do poprawnego działania aplikacji w warstwie sprzętowej:

\begin{verbatim}
pip install UniversalGPIO
\end{verbatim}

Tak przygotowane środowisko umożliwiło uruchomienie aplikacji w systemie OpenWrt bez konieczności jego rekompilacji.

Skrypt zawierający wszystkie polecenia wymienione w tym podrozdziale został dołączony wraz z raportem. Jego nazwa to \texttt{server\_install}.

\subsection{Przeniesienie kodu źródłowego}

Skompresowany plik z całym kodem źródłowym został przeniesiony na urządzenie docelowe przy pomocy polecenia \texttt{wget}.

\begin{figure}[H]
    \dirtree{%
    .1 /server.
    .2 templates/.
    .3 index.html.
    .3 login.html.
    .3 login\_failed.html.
    .2 handlers/.
    .3 base.py.
    .3 auth.py.
    .3 \_\_init\_\_.py.
    .2 /logs.
    .3 events.log.
    .2 config.py.
    .2 models.py.
    .2 enums.py.
    .2 dtos.py.
    .2 notifications.py.
    .2 messages.py.
    .2 hardware.py.
    .2 storage.py.
    .2 orders.json.
    .2 routes.py.
    .2 server.py.
    .2 client.py.
    }
    \caption{Struktura plików w projekcie \texttt{server}}
    \label{fig:tree}
\end{figure}

Folder \texttt{server} z całym kodem źródłowym został umieszczony w folderze \texttt{/usr}. Skryptom \texttt{server.py} oraz \texttt{client.py} nadane zostały uprawnienia do wykonywania:

\begin{verbatim}
    chmod +x server.py client.py
\end{verbatim}

Następnie, w folderze \texttt{/etc/init.d} umieszczony został zmodyfikowany skrypt \texttt{S99server}, uruchamiający serwer podczas startu systemu. Został on przystosowany do środowiska OpenWrt. Skrypt ten został przesłany wraz z raportem.

Ważnymi poleceniami dla poprawnego działania skryptu były:

\begin{verbatim}
    chmod +x /etc/init.d/S99server 
    /etc/init.d/S99server enable
\end{verbatim}

Poprawność działania skryptu można było przetestować manualnie w następujący sposób:

\begin{verbatim}
    /etc/init.d/S99server start 
    /etc/init.d/S99server stop 
\end{verbatim}

\subsection{Problemy z biblioteką do obsługi GPIO}

W podrozdziale \ref{konfiguracja} poleceniem \texttt{pip install UniversalGPIO} pobrana została biblioteka \texttt{UniversalGPIO}. Nie był to jednak oczywisty wybór z racji na fakt, że w poprzedniej implementacji projektu, wykorzystywana była biblioteka \texttt{pigpio} oraz związany z nią daemon \texttt{pigpiod}.

Z powodu problemów z konfiguracją daemona na systemie docelowym, zdecydowałem się na zmianę implementacji klasy \texttt{GPIOModule} tak, aby wykorzystywała inną bibliotekę do obsługi przycisków oraz diód. Dzięki zastosowaniu interfejsu \texttt{HardwareModule}, który stanowił abstrakcję na operacje wykonywane na GPIO, zmiany w projekcie ograniczyły się tylko do tej pojedynczej klasy.

\section{Podsumowanie}

Ćwiczenie laboratoryjne miało na celu praktyczne zapoznanie się z procesem uruchamiania aplikacji w systemie OpenWrt z wykorzystaniem gotowego, prekompilowanego obrazu. Przeniesienie wcześniej stworzonego rozwiązania — Systemu Kolejkowania Zleceń Produkcyjnych — do nowego środowiska pozwoliło na pogłębienie wiedzy z zakresu konfiguracji systemu, instalacji zależności oraz integracji aplikacji z mechanizmem autostartu w OpenWrt.

Ćwiczenie miało przede wszystkim charakter edukacyjny i pozwoliło na zdobycie cennych doświadczeń związanych z praktycznym zastosowaniem systemu OpenWrt w kontekście systemów wbudowanych. 

Podczas realizacji zadania napotkano jednak pewne trudności techniczne, w szczególności:

\begin{itemize}
    \item Problemy z uruchomieniem i konfiguracją demona \texttt{pigpiod}, niezbędnego do działania wcześniej wykorzystywanej biblioteki \texttt{pigpio}, co uniemożliwiło jej zastosowanie w środowisku OpenWrt.
    \item Konieczność dostosowania aplikacji do nowej biblioteki obsługującej GPIO — \texttt{UniversalGPIO}. Dzięki zastosowanej architekturze z warstwą abstrakcji sprzętowej (\texttt{HardwareModule}) modyfikacje ograniczyły się jedynie do jednej klasy.
    \item Ręczna konfiguracja sieci oraz systemowych usług startowych (takich jak skrypt \texttt{S99server}), wymagająca dobrej znajomości struktury plików i mechanizmów OpenWrt.
\end{itemize}

Pomimo tych trudności, udało się uruchomić aplikację w docelowym środowisku i zapewnić jej pełną funkcjonalność. Ćwiczenie było wartościowe zarówno z perspektywy technicznej, jak i dydaktycznej.


%----------------------------------------------------------------------
% bibliografia:

\begin{thebibliography}{9}
\bibitem{lecture} dr hab. inż. Wojciech Zabołotny, prezentacja do \textit{Wykładu 8.}
\bibitem{openwrt} Oficjalna strona środowiska OpenWrt -

\textit{https://downloads.openwrt.org/releases/24.10.1/targets/bcm27xx/bcm2711/}, dostęp \today
\bibitem{tornado} Biblioteka \texttt{Tornado} - \textit{https://www.tornadoweb.org/en/stable/}, dostęp \today
\bibitem{universalgpio} Biblioteka \texttt{UniversalGPIO} - \textit{https://pypi.org/project/UniversalGPIO/}, dostęp \today
\end{thebibliography}
%----------------------------------------------------------------------
\end{document}