\documentclass{article}
%----------------------------------------------------------------------
\usepackage{graphicx}
\usepackage[T1]{fontenc}
\usepackage[polish]{babel}
\usepackage[utf8]{inputenc}
\usepackage{amsmath}
\usepackage{multirow}
\usepackage{geometry}
\usepackage{float}
\usepackage{siunitx}
\usepackage{enumitem}
\usepackage{dirtree}
\usepackage{verbatim}
%----------------------------------------------------------------------
% strona tytułowa:

\begin{document}
\begin{titlepage}
    \centering
    \vfill    
    {\fontsize{40}{20}\selectfont \textbf{Linux w systemach wbudowanych} \par}
    \vspace{2cm}
    {\fontsize{30}{20}\selectfont Laboratorium L3 G1\par}
    \vspace{2cm}
    {\fontsize{20}{20}\selectfont Sprawozdanie\par}
    \vfill
    {\fontsize{10}{20}\selectfont Adam Grącikowski, 327350\par}
    \vspace{1cm}
    Warszawa, \today
\end{titlepage}


%----------------------------------------------------------------------
% spis treści:

\tableofcontents
\newpage
%----------------------------------------------------------------------
\section{Cel ćwiczenia laboratoryjnego}

Celem ćwiczenia jest:

\begin{itemize}
    \item zapoznanie się z bootloader’em \texttt{U-Boot},
    \item implementacja prostego serwera uruchamianego na platformie Raspberry Pi.
\end{itemize}

\section{Wymagania}

\begin{itemize}
    \item System administratora \texttt{Admin} powinien:
    \begin{itemize}
        \item Pracować z \texttt{initramfs}.
        \item Zawierać narzędzia do zarządzania kartą SD (partycjonowanie, formatowanie, kopiowanie poprzez sieć, naprawianie systemu plików itp.)
    \end{itemize}
    \item Karta SD powinna zostać podzielona an 3 partycje:
    \begin{itemize}
        \item \texttt{VFAT} z systemem ratunkowym (w katalogu \texttt{rescue}), oraz administracyjnym i jądrem systemu
 użytkowego (w katalogu \texttt{user}).
 \item \texttt{ext4} z systemem plików systemu użytkowego.
 \item \texttt{ext4} z danymi systemu użytkowego.
    \end{itemize}
    \item System użytkowy \texttt{User} powinien:
    \begin{itemize}
        \item Pracować z systemem plików \texttt{ext4} na drugiej partycji.
        \item Zawierać serwer WWW, zrealizowany przy pomocy środowiska \texttt{Tornado} języka \texttt{Python}:
        \begin{itemize}
            \item udostępniający pliki z partycji 3 na karcie SD - wyświetlający listę tych plików,
            \item pozwalający na wybranie pliku do pobrania,
            \item umożliwiający uwierzytelnionym użytkownikom wgrywanie nowych plików na tę partycję.
        \end{itemize} 
    \end{itemize}
    \item Bootloader powinien umożliwiać określenie, który system (\texttt{Admin} czy \texttt{User}) ma zostać załadowany. Działanie bootloadera powinno być sygnalizowane w następujący sposób:
    \begin{itemize}
        \item Żółta dioda powinna sygnalizować, że za chwilę sprawdzony zostanie stan przycisku.
        \item Po sekundzie odczytywany jest stan przycisku. Jeżeli nie jest wciśnięty, powinien zostać załadowany system \texttt{User}.
        \item Po sprawdzeniu przycisku żółta dioda powinna zostać zgaszona. Jeżeli został wybrany system \texttt{User}, powinna zostać zapalona zielona dioda, a jeżeli został wybrany system \texttt{Admin}, powinna zostać zpalona dioda czrewona.
    \end{itemize}
\end{itemize}

\section{Opis modyfikacji i konfiguracji Buildroota}

\subsection{Wspólna konfiguracja dla systemu \texttt{Admin} i \texttt{User}}

\begin{enumerate}[label=\arabic*.]
    \item Pobranie środowiska Buildroot, rozpakowanie oraz dokonanie wstępnej konfiguracji poprzez: \texttt{make raspberrypi4\_64\_defconfig}.
    \item \texttt{System configuration > Run a getty (login prompt) after boot > TTY port} ustawione na \texttt{console}.
    \item \texttt{Build options > Mirrors and Download locations > Primary download site} ustawione na \texttt{http://192.168.137.24/dl}.
    \item \texttt{Toolchain > Toolchain type} ustawione na \texttt{External toolchain}.
    \item \texttt{Build options > Enable compiler cache} zaznaczone i \texttt{Compiler cache location} ustawione na \texttt{/malina/gracikowskia/ccache-br}.
\end{enumerate}

\subsection{Konfiguracja dla systemu \texttt{Admin}}

\begin{enumerate}[label=\arabic*.]
    \item \texttt{Filesystem images} zaznaczona opcja \texttt{initial RAM filesystem linked into linux kernel} oraz \texttt{Compression method} ustawione na \texttt{gzip}.
    \item \texttt{System configuration > Enable root login and password > Root password} ustawione na przykładowe hasło \texttt{admin,123}.
    \item \texttt{System configuration > System banner} ustawione na \texttt{Welcome Admin}.
    \item \texttt{Target packages > BusyBox} zaznaczona opcja \texttt{Show packages that are also provided by busybox}.
    \item \texttt{Target packages > Networking applications} zaznaczona opcja \texttt{dhcp (ISC)} oraz \texttt{dhcp client}, a także \texttt{netplug}, \texttt{ntp} i \texttt{ntpd}.
    \item \texttt{System configuration > System hostname} ustawione na \texttt{admin}.
    \item \texttt{Target packages > Filesystem and flash utilities} zaznaczona opcja \texttt{dosfstools}, a w niej opcje \texttt{fatlabel}, \texttt{fsck.fat} oraz \texttt{mkfs.fat}.Dodatkowo zaznaczona opcja \texttt{e2fsprogs}, a w niej \texttt{fsck} oraz \texttt{resize2fs}.
    \item \texttt{Filesystem images > ext2/3/4 root filesystem} zaznaczona opcja \texttt{additional mke2fs options}.
    \item \texttt{Target packages > System tools > util-linux} zaznaczona opcja \texttt{mount/umount}
    \item \texttt{Target packages > Libraries > Javascript > flot} zaznaczona opcja \texttt{resize}.
    \texttt{Target packages > Hardware handling > u-boot tools} zaznaczona opcja \texttt{mkimage}.
    \item \texttt{Host utilities} zaznaczona opcja \texttt{host imx-mkimage}.
    \item \texttt{Bootloaders} zaznaczona opcja \texttt{U-Boot} oraz \texttt{Board defconfig} ustawione na \texttt{rpi\_4}.
\end{enumerate}

\subsection{Konfiguracja dla systemu \texttt{User}}

\begin{enumerate}[label=\arabic*.]
    \item \texttt{System configuration > Enable root login and password > Root password} ustawione na przykładowe hasło \texttt{user,123}.
    \item \texttt{System configuration > System banner} ustawione na \texttt{Welcome User}.
    \item \texttt{System configuration > System hostname} ustawione na \texttt{user}.
    \item \texttt{Filesystem images > ext2/3/4 variant} ustawione na \texttt{ext4}.
    \item W pliku \texttt{board/raspberrypi4-64/genimage.cfg.in} parametr \texttt{size} ustawiony na \texttt{256M}.
    \item \texttt{System configuration > Root filesystem overlay directories} ustawione na \texttt{board/overlay}.
    \item \texttt{Target packages > Networking applications} zaznaczona opcja \texttt{dhcp (ISC)} oraz \texttt{dhcp client}, a także \texttt{netplug}, \texttt{ntp} i \texttt{ntpd}.
    \item \texttt{Target packages > System tools} zaznaczona opcja \texttt{start-stop daemon}.
    \item \texttt{Target packages > Interpreter languages and scripting} zaznaczona opcja \texttt{python3} oraz w \texttt{External python modules} zaznaczone opcje \texttt{python-tornado} i \texttt{python-urllib3}.
\end{enumerate}

\section{Opis plików konfiguracyjnych i skryptów dołączonych do archiwum}

\subsection{\textit{Overlay} na system \texttt{User}}

Rysunek \ref{fig:tree-overlay} przedstawia strukturę plików dodanych w ramach nakładki (ang. \textit{overlay}) na system \texttt{User}.

\begin{figure}[H]
    \dirtree{%
    .1 /board.
    .2 overlay/.
    .3 usr/.
    .4 file-server/.
    .5 ....
    .5 server.py.
    .3 etc/.
    .4 init.d/.
    .5 S98mount.
    .5 S99server.
    }
    \caption{Struktura plików nakładce \texttt{overlay}}
    \label{fig:tree-overlay}
\end{figure}

Folder \texttt{file-server} zostanie omówiony w rozdziale \ref{implementacja-www}. Najważniejszym jego elementem jest jednak plik \texttt{server.py}, zaznaczony na rysunku \ref{fig:tree-overlay}, do którego odwołuje się skrypt \texttt{S99server}. 

Skrypt ten jest odpowiedzialny za zarządzanie uruchamianiem, zatrzymywaniem i monitorowaniem działania demona (w naszym przypadku serwera opartego na \texttt{Tornado}).

Skrypt rozpoczyna się od ustawienie zmiennych określających m.in. katalog, w którym znajduje się skrypt serwera, ścieżka do skrypu \texttt{server.py} oraz argumenty jego wywołania. Do zatrzymywania procesu został wybrany sygnał \texttt{SIGINT}.

Skrypt \texttt{S99server} zawiera funkcje pomocnicze \texttt{do\_start()}, \texttt{do\_stop()} oraz \texttt{do\_status()}, przy pomocy których odpowiednio proces uruchamiany jest w tle, proces jest zatrzymywany i wysyłany jest do niego sygnał przerwania oraz sprawdzany i wypisywany jest jego status.

Skrypt \texttt{S98mount} montuje trzecią partycję z urządzenia \texttt{mmcblk0} do katalogu \texttt{/mnt}.

\subsection{Logika Bootloader-a}

Dynamiczny wybór trybu pracy (\texttt{Admin}/\texttt{User}) został osiągnięty za pomocą jednego przycisku i wizualne potwierdzenie diodami. Logika ta znajduje się w pliku \texttt{boot.txt}.

Skrypt znajdujący się w pliku \texttt{boot.txt} został zmodyfikowany poleceniem:

\begin{verbatim}
mkimage -T script -C none -n 'Start script' -d boot.txt boot.scr
\end{verbatim}

\section{Opis implementacji serwera WWW}
\label{implementacja-www}

Implementacja logiki serwera WWW została umieszczona w projekcie \texttt{file-server} o strukturze przedstawionej na rysunku \ref{fig:tree}.


\begin{figure}[H]
    \dirtree{%
    .1 /file-tree.
    .2 templates/.
    .3 index.html.
    .3 items.html.
    .3 login.html.
    .3 login\_failed.html.
    .2 handlers/.
    .3 base.py.
    .3 auth.py.
    .3 files.py.
    .3 \_\_init\_\_.py.
    .2 file\_tree.py.
    .2 routes.py.
    .2 server.py.
    }
    \caption{Struktura plików w projekcie \texttt{file-tree}}
    \label{fig:tree}
\end{figure}

Główny skrypt uruchamiany podczas startu systemu to \texttt{server.py}. Przyjmuje on dwa pozycyjne argumenty - \texttt{port} o domyślnej wartości \texttt{8888} oraz \texttt{address} o domyślnej wartości \texttt{127.0.0.1}.

Funkcja \texttt{make\_app()} jest odpowiedzialna za stworzenie i konfigurację obiektu \texttt{tornado.web.Application}. Funkcja \texttt{main()} jest natomiast odpowiedzialna za uruchomienie serwera oraz ustawienie obsługi sygnału \texttt{SIGINT} oraz \texttt{SIGTERM} w celu ładnego zakończenia działania serwera w przypadku nadejścia sygnału.

W pliku \texttt{routes.py} znajdują się stałe określające obsługiwane ścieżki API serwera (klasa \texttt{Routes}).

W pliku \texttt{file\_tree.py} znajduje się funkcja \texttt{build\_file\_tree()} odpowiedzialna za rekurencyjne zbudowanie drzewa plików partycji.

W folderze \texttt{handlers} znajdują się implementacja odpowiednio logiki prostej autoryzacji użytkownika oraz operacji na plikach.

Zawartość wszystkich plików wymienionych na rysunku \ref{fig:tree} została udostępniona wraz ze sprawozdaniem.

%----------------------------------------------------------------------
% bibliografia:

\begin{thebibliography}{9}
\bibitem{lecture} dr hab. inż. Wojciech Zabołotny, prezentacja do \textit{Wykładu 6.}
\bibitem{tornado} Biblioteka \texttt{Tornado} - \textit{https://www.tornadoweb.org/en/stable/}, dostęp \today
\bibitem{daemon} Szkielet skryptu zarządzającego serwerem - \textit{https://gist.github.com/mrowe/8b617a8b12a6248d48b8}, dostęp \today
\bibitem{linuxiarze} Podział i formatowanie dysku w trybie tekstowym - \textit{https://linuxiarze.pl/partycje4/}, dostęp \today
\bibitem{resize2fs} Resize2fs Command - \textit{https://www.thegeekdiary.com/resize2fs-command-examples-in-linux/}, dostęp \today
\end{thebibliography}
%----------------------------------------------------------------------
\end{document}