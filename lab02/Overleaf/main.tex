\documentclass{article}
%----------------------------------------------------------------------
\usepackage{graphicx}
\usepackage[T1]{fontenc}
\usepackage[polish]{babel}
\usepackage[utf8]{inputenc}
\usepackage{amsmath}
\usepackage{multirow}
\usepackage{geometry}
\usepackage{float}
\usepackage{siunitx}
\usepackage{enumitem}
\usepackage{dirtree}
\usepackage{verbatim}
%----------------------------------------------------------------------
% strona tytułowa:

\begin{document}
\begin{titlepage}
    \centering
    \vfill    
    {\fontsize{40}{20}\selectfont \textbf{Linux w systemach wbudowanych} \par}
    \vspace{2cm}
    {\fontsize{30}{20}\selectfont Laboratorium L2 G1\par}
    \vspace{2cm}
    {\fontsize{20}{20}\selectfont Sprawozdanie\par}
    \vfill
    {\fontsize{10}{20}\selectfont Adam Grącikowski, 327350\par}
    \vspace{1cm}
    Warszawa, \today
\end{titlepage}


%----------------------------------------------------------------------
% spis treści:

\tableofcontents
\newpage
%----------------------------------------------------------------------
\section{Cel ćwiczenia laboratoryjnego}

Celem ćwiczenia laboratoryjnego jest zapoznanie się z:

\begin{itemize}
    \item procesem implementacji własnego pakietu środowiska Buildroot,
    \item obsługą przycisków i diod LED.
\end{itemize}

\section{Wymagania}

\begin{enumerate}[label=\arabic*.]
    \item Implementacja aplikacji w języku kompilowanym, obsługującej przyciski i diody LED.
    \item Przetestowanie korzystania z debugger'a przy uruchamianiu aplikacji.
    \item Przekształcenie aplikacji w pakiet Buildroot'a.
\end{enumerate}

\section{Opis aplikacji}

Aplikacja realizuje prosty symulator alfabetu Morse'a z interfejsem opartym o trzy przyciski i jedną diodę LED.

\subsection{Funkcjonalności przycisków}

Funkcjonalności przycisków są następujące:

\begin{itemize}
    \item Przycisk \texttt{DOT} - po jego naciśnięciu do sekwencji sygnałów dodawany jest krótki sygnał (kropka),
    \item Przycisk \texttt{DASH} - po jego naciśnięciu do sekwencji sygnałów dodawany jest długi sygnał (pauza, czyli myślnik),
    \item Przycisk \texttt{ACCEPT} - po jego naciśnięciu, użytkownik zatwierdza sekwencję, która następnie zostaje odtworzona przy użyciu diody LED.
\end{itemize}

Program kończy działanie po wyświetleniu całej wprowadzonej sekwencji sygnałów.

\subsection{Działanie wyświetlania}

\begin{itemize}
    \item Po zatwierdzeniu sekwencji, system iteruje po wprowadzonych sygnałach.
    \item Dla kropki dioda zapala się na krótki czas ($400ms$).
    \item Dla myślnika dioda zapala się na dłuższy czas ($800ms$).
    \item Pomiędzy kolejnymi sygnałami następuje krótkie wyłączenie ($200ms$).
\end{itemize}

\section{Struktura pakietu \texttt{morse}}

W celu utworzenia pakietu środowiska Buildroot utworzono w folderze \texttt{/package} nowy folder \texttt{/morse}. W folderze znalazły się pliki zgodnie ze strukturą przedstawioną na rysunku \ref{fig:tree}:

\begin{figure}[H]
    \dirtree{%
    .1 /morse.
    .2 src/.
    .3 Makefile.
    .3 morse.h.
    .3 morse.c.
    .3 main.c.
    .2 morse.mk.
    .2 Config.in.
    }
    \caption{Struktura plików w pakiecie}
    \label{fig:tree}
\end{figure}

Plik \texttt{Config.in} zawiera informacje o nazwie pakietu, jego krótki opis oraz informację o tym, że zależy on od pakietu \texttt{c-periphery}. 

Plik \texttt{morse.mk} określa w jaki sposób mają być pobierane źródła do pakietu oraz gdzie mają być umieszczane w docelowym systemie.

W folderze \texttt{/src} znajduje się kod źródłowy pakietu wraz ze standardowym plikiem \texttt{Makefile}.

Aby pakiet był widoczny z poziomu narzędzia \texttt{nconfig} w pliku \texttt{package/Config.in} należy dodać następujący wpis:

\begin{verbatim}
menu "LINSW custom packages"
    source "package/morse/Config.in"
endmenu
\end{verbatim}

który powoduje, że pakiet \texttt{morse} pojawia się w osobnym menu o nazwie \texttt{LINSW custom packages}.

Zawartość plików pokazanych na rysunku \ref{fig:tree} została przesłana wraz ze sprawozdaniem.

\section{Opis modyfikacji i konfiguracji Buildroota}

\begin{enumerate}[label=\arabic*.]
    \item Pobranie środowiska Buildroot, rozpakowanie oraz dokonanie wstępnej konfiguracji poprzez: \texttt{make raspberrypi4\_64\_defconfig}.
    \item \texttt{System configuration > Run a getty (login prompt) after boot > TTY port} ustawione na \texttt{console}.
    \item \texttt{Build options > Mirrors and Download locations > Primary download site} ustawione na \texttt{http://192.168.137.24/dl}.
    \item \texttt{Toolchain > Toolchain type} ustawione na \texttt{External toolchain}.
    \item \texttt{Build options > Enable compiler cache} zaznaczone i \texttt{Compiler cache location} ustawione na \texttt{/home/adam/linsw/ccache-br}.
    \item \texttt{Filesystem images} zaznaczona opcja \texttt{initial RAM filesystem 
linked into linux kernel} oraz \texttt{Compression method} ustawione na \texttt{gzip}.
    \item W pliku \texttt{board/raspberrypi4-64/genimage.cfg.in} parametr \texttt{size} ustawiony na \texttt{128M}.
    \item \texttt{System configuration > Enable root login and password > Root password} ustawione na przykładowe hasło \texttt{root,123}.
    \item \texttt{System configuration > System banner} ustawione na \texttt{Welcome Adam Gracikowski}.
    \item \texttt{System configuration > Root filesystem overlay directories} ustawione na \texttt{board/overlay}.
    \item \texttt{Host utilities} zaznaczona opcja \texttt{host environment-setup}.
    \item \texttt{LINSW custom packages} zaznaczona opcja \texttt{morse}.
\end{enumerate}

\section{Opis implementacji logiki aplikacji}

Punktem startowym aplikacji jest funkcja \texttt{main} znajdująca się w pliku \texttt{main.c}. Inicjalizacja obiektów GPIO (diody oraz trzech przycisków) została zrealizowana przy pomocy funkcji \texttt{gpio\_new()}, \texttt{gpio\_open()} oraz w przypadku przycisków dodatkowo \texttt{gpio\_set\_edge()}, które ustawia nasłuchiwanie na zdarzenia określone flagą \texttt{GPIO\_EDGE\_FALLING}.

Przechowywanie sekwencji sygnałów wprowadzonej przez użytkownika zostało zrealizowane przy pomocy prostej listy powiązanej, aby zapewnić nieograniczoną z góry długość sekwencji.

Następnie program wchodzi w główną pętlę, w której oczekiwanie na zdarzenia przycisków zrealizowane zostało przy pomocy funkcji \texttt{gpio\_pool\_multiple()}. Funkcja ta przyjmuje jako parametry wywołania tablicę uchwytów do obiektów GPIO, długość tej tablicy, maksymalny czas oczekiwania na zdarzenie (w tym przypadku ustawiony na wartość $-1$, oznaczającą nieskończone oczekiwanie), a także tablicę typu \texttt{bool}, do której zostanie zapisana informacja, dla których przycisków wystąpiło zdarzenie.

W dalszej części pętli wykonywana jest warunkowo logika odpowiedzi na wciśnięcie poszczególnych przycisków. 

Ważnym elementem pętli jest wyjmowanie z kolejki zdarzeń przy pomocy \texttt{gpio\_read\_event()}. Po wywołaniu tej funkcji następuje sprawdzenie, czy rozważane zdarzenie nie jest przypadkiem efektem zjawiska drgania styków (ang. \textit{switch bounce}). Jeżeli zdarzenie wystąpiło zbyt szybko w porównaniu do ostatnio zarejestrowanego, to jest ono ignorowane. Sprawdzenie to odbywa się przy pomocy prostej funkcji \texttt{has\_miliseconds\_passed()} oraz tablicy \texttt{timestamps} typu \texttt{uint64\_t}, w której przechowywane są czasy ostatnio zarejestrowanych zdarzeń. Następuje również czyszczenie kolejki dla odpowiedniego przycisku przy pomocy funkcji \texttt{empty\_queue()}.

Po wyjściu z pętli następuje sekcja \texttt{cleanup}, odpowiedzialna za zwalnianie oraz zamykanie zasobów GPIO, a także zwalnianie zaalokowanej pamięci.

%----------------------------------------------------------------------
% bibliografia:

\begin{thebibliography}{9}
\bibitem{embeddedinn} Adding Custom Packages to Buildroot - \textit{https://embeddedinn.com/articles/tutorial/Adding-Custom-Packages-to-Buildroot}, dostęp: \today
\bibitem{buildroot} Adding New Packages to Buildroot - \textit{https://buildroot.org/downloads/manual/manual.html}, dostęp \today
\end{thebibliography}
%----------------------------------------------------------------------
\end{document}